\documentclass[12pt]{article}
\usepackage[utf8]{inputenc}
\usepackage[T2A]{fontenc}
\usepackage[english]{babel}
\usepackage{amsmath,amssymb,amsthm,amsfonts}
\usepackage[top=2cm,bottom=2cm,left=2cm,right=2cm]{geometry}
\usepackage{secdot}
\usepackage{lineno}



\title{Solution of the home assignment  № 3}
\author{Vladimir Saraikin}
\date{}

\begin{document}

\maketitle

\section{Problem 1}

2. Any strongly convex function has a unique global minimum 
\[ \text{Strongly convex functions, by definition, have a unique global minimum due to the strictness of their convexity.} \]


\section{Problem 2}

\begin{enumerate}
    \item $ f(x) = \sup_{y \in C} \langle y, x \rangle $: Convex. The supremum of a set of linear functions is convex.
    \item $ f(x) = \| Ax - b \| $: Convex. Norms are convex, and the composition of a convex function with an affine function is convex.
    \item $ f(x) = \sum_{i=1}^{n} |x_i|^{1/2} $: Neither convex nor concave. The function \( |x|^{1/2} \) is not globally concave or convex.
    \item $ f(X) = \lambda_1(X) + \ldots + \lambda_k(X) $: Convex. The sum of the largest \( k \) eigenvalues of a symmetric positive semidefinite matrix is convex.
    \item $ f(x) = \min_{i=1,\ldots,n} x_i $: Convex. The minimum of a set of linear functions is convex.
    \item $ f(x) = - \left( \sum_{i=1}^{n} x_i \right)^{1/n} $: Concave on \( \mathbb{R}^n_+ \). The geometric mean is concave, and multiplying by -1 reverses convexity/concavity.
    \item $ f(w) = \sum_{i=1}^{m} \log(1 + e^{-y_i \langle w, x_i \rangle}) + \frac{1}{2} \| w \|_2^2 $: Convex. The log-sum-exp function and the quadratic function are convex.
    \item $ f(X, Y) = \| A - XY \|_F $: Convex. The Frobenius norm is convex, and the function is affine in \( X \) and \( Y \).
    \item $ f(W_1, W_2) = \| W_1 \max(W_2 x, 0) \|_2^2 $: Convex. The composition of a convex non-decreasing function with a convex function is convex.
\end{enumerate}


\section{Problem 3}

\begin{enumerate}
    \item Lipschitz constant of gradient bounds from above the norm of Hessian: \\
    Correct. If a function \( f \) has a gradient \( \nabla f \) that is Lipschitz continuous with constant \( L \), then \( \| \nabla^2 f(x) \| \leq L \). This follows from the definition of Lipschitz continuity for the gradient.

    \item Lipschitz constant of gradient bounds from above the absolute values of function: \\
    Incorrect. The Lipschitz constant of the gradient of a function does not directly bound the absolute values of the function itself. It relates to the rate of change of the gradient.

    \item Lipschitz constant of function bounds from above the norm of Hessian: \\
    Incorrect. The Lipschitz constant of a function pertains to the function's values and does not provide information about the norm of the Hessian.

    \item Lipschitz constant of function bounds from above the norm of gradient: \\
    Correct. If a function \( f \) is Lipschitz continuous with constant \( K \), then \( \| \nabla f(x) \| \leq K \). This is derived from the mean value theorem and the definition of Lipschitz continuity for the function.
\end{enumerate}


\end{document}
